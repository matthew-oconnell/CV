\documentclass[margin,line]{res}


\oddsidemargin -.5in
\evensidemargin -.5in
\textwidth=6.0in
\itemsep=0in
\parsep=0in

\newenvironment{list1}{
  \begin{list}{\ding{113}}{%
      \setlength{\itemsep}{0in}
      \setlength{\parsep}{0in} \setlength{\parskip}{0in}
      \setlength{\topsep}{0in} \setlength{\partopsep}{0in} 
      \setlength{\leftmargin}{0.17in}}}{\end{list}}
\newenvironment{list2}{
  \begin{list}{$\bullet$}{%
      \setlength{\itemsep}{0in}
      \setlength{\parsep}{0in} \setlength{\parskip}{0in}
      \setlength{\topsep}{0in} \setlength{\partopsep}{0in} 
      \setlength{\leftmargin}{0.2in}}}{\end{list}}


\begin{document}

\name{Matthew D. O'Connell \vspace*{.1in}}

\begin{resume}
\section{\sc Contact Information}
\vspace{.05in}
\begin{tabular}{@{}p{2in}p{4in}}
                  
        
University of Tennessee at  & {\it Voice:}  (931) 561-3848 \\        
Chattanooga: SimCenter &  {\it E-mail:}  Matthew-OConnell@mocs.utc.edu\\   
Chattanooga, TN  37403 USA  & \ \ \ \ \ \ \ \ \ \ \ Matthew.David.OConnell@gmail.com\\     
\end{tabular}

\section{\sc Brief}
\begin{list2}
\item 2013, Pathways IEP at NASA Langley Research Center. 
\item 2011, Master's in Computational Engineering University of Tennessee at Chattanooga. GPA 4.0.
\item 2011, Began PhD in Computational Engineering at the University of Tennessee at \\ Chattanooga SimCenter: National Center for Computational Engineering.
\item Anticipated Graduation: May 2015.
\item Research Interests: mesh generation, mesh optimization, feature based mesh adaptation, elliptic smoothing on unstructured meshes, mesh generation for massively parallel architectures, \\ polyhedral mesh generation
\end{list2}



\section{\sc Education}
{\bf University of Tennessee at Chattanooga}, Chattanooga, Tennessee USA\\
\vspace*{-.1in}
\begin{list1}
\item[] Ph.D. in Computational Engineering, expected May 2015
\vspace*{.05in}
\item[] M.S. in Computational Engineering,  August 2011
\end{list1}

{\bf Austin Peay State University}, Clarksville, Tennessee USA\\
\vspace*{-.1in}
\begin{list1}
\item[] B.S., Physics,  May, 2009
\end{list1}

\section{\sc Conference Presentations}

O'Connell, Matthew D. and Karman, Steve L. ``Techniques for Unstructured Mesh Adaptation with Elliptic Smoothing''.  $50^{th}$ {\em American Institute of Aeronautics and Astronautics} Aerospace Sciences Meeting.

O'Connell, Matthew D. and Karman, Steve L. ``Mesh Rupturing: A Technique for Significant Mesh Movement''. $51^{st}$ {\em American Institute of Aeronautics and Astronautics} Aerospace Sciences Meeting.


\section{\sc Selected Courses}
\begin{list2}
\item Grid Generation  
\item Adaptive and Dynamic Grid Generation 
\item Parallel Scientific Supercomputing
\item Computational Fluid Dynamics 
\item Viscous Flow Theory
\item Viscous Flow Computation
\item Computational Structural Dynamics 
\item Computational Design
\item Numerical Solutions of Partial Differential Equations
\item Numerical Analysis 
\end{list2}

%\begin{list2}
%\vspace*{.05in}
%\item Thesis Topic: ``Comparison of Two Methods for Two Dimensional Unstructured Mesh Adaptation with Elliptic Smoothing''
%\end{list2}





\section{\sc Academic and Research Experience}

{\bf University of Tennessee at Chattanooga SimCenter}\\
{\em Graduate Student} with Steve Karman \hfill {\bf August, 2009 - present}\\
Includes current Ph.D.~research, Ph.D.~and Masters level coursework and
research/consulting projects.  Current research includes three dimensional Winslow / elliptic unstructured mesh smoothing under Dr. Steve Karman.  Past research included two dimensional Winslow / elliptic smoothing and feature based mesh adaptation.  Course work included: implementing two and three dimensional unstructured CFD codes, structural response codes, distributed and shared memory parallelization of numerical algorithms, mesh generation and smoothing, and coupling mesh and CFD codes to implement Adjoint and gradient based geometry design optimization.

%\vspace{-.1cm}
{\bf NASA Real World in World}\\
{\em Evaluator} \hfill {\bf February - April, 2011}\\
Evaluate student design of James Webb Space Telescope.  Work with student teams to clarify and justify their designs for a deployable sunshield and mirror assembly.

{\bf Austin Peay State University}\\
{\em Mentor for Governor's School for Computational Physics} \hfill {\bf June - July, 2008 \& 2009}\\
\begin{itemize}
\item Lead mentor - managed schedules and activities in and outside the classroom of other mentors.
\item Teaching assistant - shared administrative responsibilities with faculty
instructor, fielding student inquiries, holding bi-weekly recitation sessions, small group and one-on-one tutoring, teaching computational and experimental labs, developed projects and exams.
\end{itemize}

{\bf Austin Peay State University}\\
{\em Teaching Assistant: Astronomy 1010 Lab} \hfill {\bf May - July, 2007}\\
Maintain lab equipment, field student inquiries, grade lab reports. 

{\bf Austin Peay State University}\\
{\em Tutor: Department of Physics} \hfill {\bf May - July, 2007}\\
Teach weekly recitation sessions for freshman Physics Majors in Calculus and Calculus based Physics courses. 


%\section{\sc Publications}

%\section{\sc Papers in preparation}


%\section{\sc Professional Experience}
%{\bf Bureau of Transportation Statistics, U.S. Department of
%  Transportation}, Washington, District of Columbia USA
%
%\vspace{-.3cm}
%{\em Summer researcher} \hfill {\bf May, 2000 - August, 2000}\\
%Carried out several consulting projects, including modelling of
%injuries to cadavers in crash test experiments, analysis of airline
%delay data, and advice on analysis of airline economics data.

\section {\sc Technical Experience}
\begin{list2}
	\item Recent Regular Use: C/ C++, MPI, Octave, Matlab
	\item Past Regular Use: Fortran 90, Python, OpenMP, POSIX Threads, 
	\item Familiar: OpenCL, Qt, Java, Fortran 77, PHP
	\item Applications: Mathematica, VisIt, Paraview, XCode, Netbeans, common spreadsheet and presentation software.  
	\item Operating Systems: Unix/ Linux, OS X, Windows
\end{list2}



\section{\sc Honors and Awards} 
Austin Peay State Univeristy: graduated Magna Cum Laude, in Physics, Sigma Pi Sigma, 2009 \\
Robert Sears Award, Dedication to Science 2009 \\
National Space Grant Recipient 2007 \\




\end{resume}
\end{document}




